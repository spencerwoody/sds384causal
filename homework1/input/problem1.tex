
\section{Problem 1}

\begin{quoting}
  The canvas site provides an excerpts from the published HEI Research
  Report 148, \emph{Impact of Improved Air Quality During the 1996
    Summer Olympic Games in Atlanta on Multiple Cardiovascular and
    Respiratory Outcomes} by Peel et al.. The first four pages are
  excerpts from an overall summary and critique of the report, which
  provide general background. The next five pages are excerpts from
  the more detailed research report. In both cases, the text appearing
  in \textcolor{red}{red boxes} should be sufficient to answer the
  following questions.
\end{quoting}

\begin{enumerate}[(a)]
\item
  \begin{quoting}
    For both the analysis of ozone concentrations and the analysis of
    ED visits, what are the potential outcomes defining the effects of
    interest?
  \end{quoting}
  This study analyzes the impact of a short-term, temporary
  intervention designed to reduce car traffic in Atlanta during the
  1996 Summer Olympic Games on (i) daily ozone concentrations in the
  city measured at a specific site, and (ii) the number of visits to
  emergency departments related to cardiovascular and respiratory
  cases for specified cohorts of interest aggregated across twelve
  hospitals in the city.

  Let $Y_i$ denote the ozone concentration in Atlanta on a single day
  $i$, $Z_i$ denote the number of ED visits on day $i$, and
  $W_i \in \{0, 1\}$ be an indicator for whether the intervention is
  applied to day $i$.  Then the causal effects of interest compare the
  two sets of potential outcomes:
  \begin{align*}
    (Y_i(W_i = 0), Y_i(W_i = 1))
    \quad
    \text{and}\quad (Z_i(W_i = 0), Z_i(W_i = 1)).
  \end{align*}
    That is, the causal effects concern the differences between ozone
  concentration when the intervention is applied versus when it is not
  applied, and likewise for ED visits.
\item
  \begin{quoting}
    For both the analysis of ozone concentrations and the analysis of
    ED visits, provide a possible violation of the ``No Multiple
    Versions of Treatment'' (or ``consistency'') part of SUTVA.
  \end{quoting}
  There are several possible violations of the consistency portion of
  the stable unit treatment value assumption (SUTVA), mainly stemming
  from the city's adherence to the intervention.  For instance, one
  provision of the intervention was to encourage businesses to provide
  telecommuting work options and alternative work hours for employees,
  and to advocate the use of vacation time.  However, the degree to
  which businesses uphold these policies could vary throughout the
  length of the Olympic Games.  There could be some days when many
  businesses do not allow for telecommuting, for instance due to need
  to produce quarterly results.  This would result in multiple
  versions of the treatment, in that the extent of the treatment would
  change.
\item
  \begin{quoting}
    For both the analysis of ozone concentrations and the analysis of
    ED visits, describe the assignment mechanism and provide one
    reason why it may not be \emph{unconfounded}.
  \end{quoting}
  The choice of when to deploy the intervention was deterministically
  chosen by the timing of the 1996 Olympics.  Any event which affects
  the measured outcomes and takes place during the Olympics could
  confound the treatment assignment.  For instance, there is likely an
  increase in plane traffic arriving at the Atlanta airport carrying spectators and
  athletes, and these planes could also increase the level of ozone in
  the air.  Also, there are expected to be many more people in the
  city for the Olympics compared to normal, and this would increase
  the amount of car traffic.  
\end{enumerate}

%%% Local Variables:
%%% mode: latex
%%% TeX-master: "../main"
%%% End:
