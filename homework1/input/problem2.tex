
\section{Problem 2}

\begin{quoting}
  Describe an example of an assignment mechanism that is ignorable but
  not unconfounded and support your argument with statement(s) about
  the assignment mechanism.
\end{quoting}

The ignorability assumption \citep{rubin1978} may be expressed as
\begin{align}\label{eq:ignorable}
  \W \ind \Y^{\mis} \mid \Y^{\obs}, \X
\end{align}

The assumption of unconfoundedness \citep{PS} can be expressed as:
\begin{align}\label{eq:unconfounded}
  \W \ind (\Y(0), \Y(1)) \mid \X.
\end{align}

% Probabilistic assignment (also called positivity):
% \begin{align}\label{eq:postivity}
%   0 < p_i(\X, \Y(0), \Y(1)) < 1
% \end{align}


We may create an assignment mechanism that is ignorable but not
unconfounded using a sequential treatment assignment that conditions
on observed outcomes.  For example, consider sampling treatment by
taking draws of balls from an urn with replacement, and we put more
balls in the urn depending on the outcome of $Y_i^\obs$.  To give a
specific example, first define $N_s^{[0]} = N_d^{[0]} = 0$.  The
outcome $Y$ is either 1, for survival, or 0, denoting death.  Let
treatment assignments for units $i=1, \ldots, N$ be determined by
iterating through the following steps:
\begin{enumerate}[(i)]
\item Calculate probability to treatment for unit $i$
  \begin{align*}
    p^{[i]} = \frac{1 + N_s^{[i-1]}}{2 + N_s^{[i-1]} + N_d^{[i-1]}}.
\end{align*}
\item Assign treatment via
  \begin{align*}
    W_i \sim \text{Bernoulli}(p^{[i]}) .
  \end{align*}
\item Observe $Y_i^\obs = Y_i(W_i) \in \{0, 1\}$
\item If unit $i$ is assigned to treatment, then update the weights
  for assignment to treatment and control depending on the outcome of
  $Y_i(W_i = 1)$.  If unit $i$ is assigned treatment and dies, give
  more weight to assignment to control for the next unit; if unit $i$
  is assigned treatment and survives, give more weight to treatment
  for the next unit.  Specifically, the weights are updated by
  \begin{align*}
    N_s^{[i]} =
    \begin{cases}
      N_s^{[i-1]} + 1 & \text{if } W_i = 1  \text{ and } Y_i = 1  \\
      N_s^{[i-1]} & \text{otherwise} \\
    \end{cases} \quad \text{and} \quad
    N_d^{[i]} =
    \begin{cases}
      N_d^{[i-1]} + 1 & \text{if } W_i = 1  \text{ and } Y_i = 0 \\
      N_d^{[i-1]} & \text{otherwise}. \\
    \end{cases}
  \end{align*}

\item If $i < N$, return to step $(i)$. 

    The treatment assignment vector $\mathbf{W}$ therefore depends on
  the observed values in $\Y^\obs$, but it is still independent of the
  missing outcomes $\Y^\mis$; therefore condition \eqref{eq:ignorable}
  holds, but condition \eqref{eq:unconfounded} does not.


\end{enumerate}


%%% Local Variables:
%%% mode: latex
%%% TeX-master: "../main"
%%% End:
