
\section{Exercise 2}

\begin{quoting}
  In this exercise you will use a variety of propensity score methods
  to estimate the causal effect of having an SnCR in a given year on
  $\text{NO}_x$ emissions in that year, under the assumption that the
  covariates listed in Table~\ref{tab:data-description} are sufficient
  to adjust for confounding (i.e., that having an SnCR installed is
  conditionally unconfounded with respect to $\text{NO}_x$
  emissions). For all parts of this exercise:
  \begin{itemize}
  \item Use logistic regression with all of the variables in
    Table~\ref{tab:data-description} (besides Tx and Outcome) included
    as covariates to estimate the propensity score.
  \item Be sure to check covariate balance for each analysis
  \item Conduct each analysis separately for 2002 and 2014, and
    comment (in $\sim$3 sentences) on the differences between the analyses
    in the two years Be sure to check covariate balance for each
    analysis
  \item I strongly suggest you read up on the following R packages to
    conduct these analyses: \texttt{MatchIt}, \texttt{survey},
    \texttt{ipw}, \texttt{twang}
  \end{itemize}
\end{quoting}

\begin{enumerate}[(a)]
\item
  % -------------------------------------------------------------------------
  % A
  % -------------------------------------------------------------------------
  \begin{quoting}
    When you arrive at a propensity score model, plot the histograms
    of the estimated propensity scores in treated and untreated units.
  \end{quoting}
  % -------------------------------------------------------------------------

\item
  % -------------------------------------------------------------------------
  % B
  % -------------------------------------------------------------------------
  \begin{quoting}
    Conduct a 1-1 nearest neighbor propensity score matching procedure
    without replacement
  \end{quoting}
  % -------------------------------------------------------------------------

\item
  % -------------------------------------------------------------------------
  % C
  % -------------------------------------------------------------------------
  \begin{quoting}
    Conduct a 1-1 nearest neighbor propensity score matching procedure
    without replacement and a caliper set to 0.1 standard deviations
    of the estimated propensity score distribution.
  \end{quoting}
  % -------------------------------------------------------------------------

\item
  % -------------------------------------------------------------------------
  % D
  % -------------------------------------------------------------------------
  \begin{quoting}
    Conduct an analysis that subclassifies units based on the
    estimated propensity score
  \end{quoting}
  % -------------------------------------------------------------------------

\item
  % -------------------------------------------------------------------------
  % E
  % -------------------------------------------------------------------------
  \begin{quoting}
    Conduct an IPW analysis using weights
    $\frac{W_i}{\hat e(X_i)} + \frac{1 - W_i}{1 - \hat e(X_i)}$ and be
    sure to include a visual summary (e.g., histogram) of the
    estimated weights.
  \end{quoting}
  % -------------------------------------------------------------------------

\item
  % -------------------------------------------------------------------------
  % F
  % -------------------------------------------------------------------------
  \begin{quoting}
    Conduct an IPW analysis using stabilized weights and be sure to
    include a visual sum- mary (e.g., histogram) of the estimated
    weights.
  \end{quoting}
  % -------------------------------------------------------------------------
  
\end{enumerate}

%%% Local Variables:
%%% mode: latex
%%% TeX-master: "../main"
%%% End:
