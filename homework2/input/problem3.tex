
\section{Exercise 3}

\begin{quoting}
  Describe in $\sim$5 sentences why the answers you obtained with the
  different propensity score methods in Exercise (2) were different
  from one another.
\end{quoting}

Every analysis in Exercise 2 used a linear model to estimate the
causal effect of SnCR installation on log-emissions of NO$_x$.
However, the estimates differed considerably.  This is all due to how
observations were either selected or weighted.  That is, matching,
both with and without a caliper, removed many observations from
consideration when estimating the parameters of the linear model in
Eq.~\eqref{eq:lin-mod} because there were not any units with
adequately similar propensity scores to these units.  Alternatively,
the IPW analyses included every observation in the analysis, but
weighted each one differently.  Finally, the subclassification
analysis also included every unit, but weighted each one according to
the size of the subclassiciation (based on the estimated propensity
score) to which it belonged.

%%% Local Variables:
%%% mode: latex
%%% TeX-master: "../main"
%%% End:
