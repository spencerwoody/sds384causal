


% -------------------------------------------------------------------------

\begin{frame}
  \frametitle{Introduction}



  Let $\mathcal{Z}$ be the sample space for the treatment assigment
  $Z$.

  \bigskip 

  \begin{itemize}
  \item Most of our course has only considered binary treatments.  
    \begin{align*}
      Z \in \mathcal{Z} = \{0, 1 \}
    \end{align*}
    Causal estimands are comparisons of counterfactual outcomes
    $Y_i(Z_i = 1)$ vs $Y_i(Z_i = 0)$ \medskip
  \item Now we consider nonbinary treatments \medskip 
    \begin{itemize}
    \item \textbf{Categorical} (possibly ordinal):
      $\mathcal{Z} = \{1, 2, \ldots, k \}$,\\  e.g.~multiple treatment
      arms \medskip 
    \item \textbf{Continuous}: $\mathcal{Z} \subseteq \mathbb{R}$,
      e.g.~drug dose \medskip 
    \end{itemize}

  \end{itemize}


  % \textcolor{grey}{Not considering time-varying treatments here\ldots}


\end{frame}

% -------------------------------------------------------------------------

\begin{frame}
  \frametitle{Causal estimands in the Rubin Causal Model}

  \textbf{Categorical treatment} with $k$ categories: \medskip

  \begin{itemize}
  \item There are $\binom{k}{2}$ pairwise comparisons of treatment assignment
    \begin{align*}
      Y_i(Z_i = j) \text{ vs. } Y_i(Z_i = {j'}) \text{ for } j,j'\in \{1,2,\ldots,k\}
    \end{align*}
  \end{itemize}
  
  \textbf{Continuous treatment}: \medskip 
  \begin{itemize}
  \item Finite difference comparison
    \begin{align*} 
      Y_i(Z_i = z) \text{ vs. } Y_i(Z_i = z') \text{ for } z \neq z'
    \end{align*}
  \item \textit{Average dose-response function}
    \begin{align*}
      \mu(z) = \E[Y_i(z)]
    \end{align*}
  \end{itemize}
  
\end{frame}

% -------------------------------------------------------------------------

\begin{frame}
  \frametitle{Generalized propensity score}
  
  Let $X$ be the vector of observed covariates. \bigskip 






  \begin{block}{\textit{Definition}: Generalized propensity score\footnote{\cite{Imbens2000,Hirano2004}} (GPS)}
    Let $r(z,x)$ be the conditional density (or mass function) of the
    treatment given the covariates:
    \begin{align*}
      r(z, x) = f_{Z \mid X}(z \mid x)
    \end{align*}
    The generalized propensity score is $R = r(Z, X)$. 
  \end{block}

  \bigskip

  Note that $R$ may be a vector, e.g. if $Z$ is categorical. 

\end{frame}

% -------------------------------------------------------------------------

\begin{frame}
  \frametitle{Overlap}
  
  \begin{block}{\textit{Assumption}: Overlap}
    $$r(z, x) = f_{Z \mid X}(z \mid x) > 0 \quad \forall z \in \mathcal{Z}$$
  \end{block}
  
\end{frame}

% -------------------------------------------------------------------------

\begin{frame}
  \frametitle{Generalized propensity score}

    \begin{block}{\textit{Assumption}: Weak unconfoundedness}
    $Y(z) \ind Z \mid X$ for all $z \in \mathcal{Z}$ \smallskip

    \footnotesize \emph{Note: this does not require joint independence of all
    potential outcomes $\{Y(z)\}_{z\in \mathcal{Z}}$} \normalsize
  \end{block}

  Similar to Rosenbaum and Rubin (1983) for the case of binary $Z$,
  \cite{Imbens2000} and \cite{Hirano2004} demonstrate:

  \begin{block}{\textit{Theorem}: Weak unconfoundedness given the GPS}
    If weak unconfoundedness holds given $X$, then, for every $z$,
    \begin{align*}
      f_{Z}(z \mid r(z, X), Y(z)) = f_{Z}(z \mid r(z, X)).
    \end{align*}

  \end{block}
  
  
\end{frame}

% -------------------------------------------------------------------------
\begin{frame}
  \frametitle{Existing methods mostly rely on GPS}
  
  \begin{itemize}
  \item \cite{imai2004}: Subclassify on GPS, then take average over
    subclasses \medskip 
  \item Hirano and Imbens (2004): Parametric model for $Y \mid Z, R$,
    then marginalize over $R$ \medskip 
  \item \cite{robins2000marginal}: IPTW estimator using GPS
    \medskip
  \end{itemize}

  \textbf{Disadvantage}: These methods rely on parametric assumptions
  \bigskip

  \emph{Work on matching for nonbinary treatments is relatively new}

\end{frame}

% -------------------------------------------------------------------------

\begin{frame}
  \frametitle{Outline}

  Presenting methodologies from three papers: \bigskip 

  \begin{enumerate}[(i)]
  \item \cite{nattino2019}: Compare treatment effects across 3
    treatment arms (\textit{categorical}) \medskip
  \item \cite{svje2017}: Generalized full matching for multiple
    treatment categories (\textit{categorical}) \medskip
  \item \cite{wu2020matching}: Use matching to estimate average
    dose-response (\textit{continuous})
  \end{enumerate}
  
\end{frame}


% -------------------------------------------------------------------------

% \begin{frame}
%   \frametitle{Other methods for continuous treatments}
  
%   \begin{itemize}
%   \item \cite{RIClinear,bac}: Bayesian linear regression  \medskip 
%   \item Woody et al. (2020+): Nonparametric regression, assuming
%     (heterogeneous) linear treatment effects \medskip
%   \item \cite{kennedy2017}: non parametric double-robust estimation of
%     average dose-response
%   \end{itemize}
  
% \end{frame}

%%% Local Variables:
%%% mode: latex
%%% TeX-master: "../main"
%%% End:


