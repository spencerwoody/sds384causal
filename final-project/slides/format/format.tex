


% Specify sans body text, math serif text
\documentclass[sans, mathserif, professionalfont, 11pt]{beamer}

% \documentclass[sans]{beamer}

% \documentclass{beamer}

% Specify fonts
% \usepackage[default]{roboto}
% \usepackage[scaled]{helvet} % Sans sefif body text
\usepackage{txfonts}
\usepackage[scaled]{helvet} % Sans sefif body text
% \usepackage[]{comicneue} % Sans sefif body text
% \usepackage{cmbright}
% \usepackage[default]{comicneue}
% \usepackage{mathpazo} % Palatino for math
% \usepackage{eulervm}
\usepackage[scaled=0.95]{inconsolata} % Inconsolata fixed width font
% \usepackage[]{inconsolata} % Inconsolata fixed width font

% \usepackage[]{txfonts}

\usepackage{tikz}
\usepackage{ragged2e}
\usepackage{lipsum}
\usepackage{hyperref}

\justifying

% Bibliography
\usepackage[round]{natbib}
% \bibliographystyle{apacite}
\usepackage{appendixnumberbeamer}
\bibliographystyle{plainnat}
% \AtBeginDocument{\renewcommand{\harvardand}{\&}}

\newcommand\ind{\protect\mathpalette{\protect\independenT}{\perp}}
\def\independenT#1#2{\mathrel{\rlap{$#1#2$}\mkern2mu{#1#2}}}


\newcommand{\U}{\mathbf{U}}
\newcommand{\M}{\mathbf{M}}
\newcommand{\m}{\mathbf{m}}
\newcommand{\w}{\mathbf{w}}

\newcommand{\N}{\mathcal{N}}
\newcommand{\dee}{\mathrm{d}}
\newcommand{\F}{\mathcal{F}}
\newcommand{\trans}{\intercal}
\newcommand{\y}{\mathbf{y}}
\newcommand{\f}{\mathbf{f}}
\newcommand{\I}{\mathcal{I}}
\newcommand{\gammatilde}{\tilde{\gamma}}
\newcommand{\betabar}{\bar{\beta}}
\newcommand{\Ytilde}{\tilde{Y}}
\newcommand{\LL}{\mathcal{L}}
\newcommand{\E}{\mathrm{E}}


% Use mathbb symbols from Computer Modern
% \AtBeginDocument{
%   \DeclareSymbolFont{AMSb}{U}{msb}{m}{n}
%   \DeclareSymbolFontAlphabet{\mathbb}{AMSb}}

\DeclareMathOperator{\EE}{E}

% Graphics, tables, math, etc.
\usepackage{graphicx} 
\usepackage{booktabs} 
\usepackage{amsmath}
\usepackage{bm}
\usepackage{amssymb}
\usepackage{xcolor}

% \usepackage[symbol]{footmisc}

% \renewcommand{\thefootnote}{\fnsymbol{footnote}}

\renewcommand*{\thefootnote}{\fnsymbol{footnote}}

% Formatting
\setbeamertemplate{caption}[numbered]

\setbeamertemplate{title page}
{
  \begin{minipage}[b][\paperheight]{\textwidth}
    \vfill
    \ifx\inserttitle\@empty%
    \else%
    {\raggedright\linespread{0.8}\usebeamerfont{title}\usebeamercolor[fg]{title}
      % \scshape\MakeLowercase{\inserttitle}\par}%
      \scshape
      {\inserttitle}\par}%
    \vspace*{0.5em}
    \fi%
    \ifx\insertsubtitle\@empty%
    \else%
    {\usebeamerfont{subtitle}\usebeamercolor[fg]{subtitle}\insertsubtitle\par}%
    \vspace*{0.5em}
    \hrule
    \fi%
    \vspace*{0.5em}
    \ifx\insertauthor\@empty%
    \else%
    {\usebeamerfont{author}\usebeamercolor[fg]{author}\insertauthor\par}%
    \vspace*{0.25em}
    \fi%
    % Institute
    \ifx\insertinstitut\@empty%
    \else%
    \vspace*{3mm}
    {\usebeamerfont{institute}\usebeamercolor[fg]{institute}\insertinstitute\par}%
    \fi%
        \vspace*{3mm}
        % Date
    \ifx\insertdate\@empty%
    \else%
    {\usebeamerfont{date}\usebeamercolor[fg]{date}\insertdate\par}%
    \fi%
    \vspace*{5mm}
    % \vfill
    % -------------------------------------------------------------------------
    % -------------------------------------------------------------------------
    % -------------------------------------------------------------------------
    % -------------------------------------------------------------------------
    % University logo; if none, comment out
    \if1\graphic{
      % \begin{center}
      \includegraphics[width=0.5\textwidth]{img/department-logo.png}\\
      % \includegraphics[width=0.6\textwidth]{img/McCombs_School_Brand_Branded.png}
      % \includegraphics[width=0.6\textwidth]{img/irom-logo.png}
      % \end{center}
    }\fi
    % -------------------------------------------------------------------------
    % -------------------------------------------------------------------------
    % -------------------------------------------------------------------------
    % -------------------------------------------------------------------------
  \vfill
  \vspace*{5mm}
  \end{minipage}
}

% \AtBeginSection{\frame{\sectionpage}}

\setbeamertemplate{frametitle}
  {\begin{flushleft}\smallskip 
    \insertframetitle\par 
   \vskip1pt\end{flushleft}}
\setbeamertemplate{itemize item}{$\bullet$}
\setbeamertemplate{navigation symbols}{}
% \setbeamertemplate{footline}[text line]{%
%     \hfill
% }
 \setbeamertemplate {footline}{\quad\hfill\insertframenumber\strut\quad}
\setbeamertemplate{footline}{
  \hfill%
  \usebeamercolor[fg]{page number in head/foot}%
  \usebeamerfont{page number in head/foot}%
  % Use this line for only the current frame number
  \setbeamertemplate{page number in head/foot}[framenumber]%
  % Use this line for frame number and *total* frame number
    % \setbeamertemplate{page number in head/foot}[totalframenumber]%
  \usebeamertemplate*{page number in head/foot}\kern1.5em\vskip8pt
}


% You can customize color settings here

% % Define some colors:

% Colors from R; see following page
% http://research.stowers.org/mcm/efg/R/Color/Chart/ColorChart.pdf
\input{format/r-colors}

% \definecolor{DarkFern}{HTML}{407428}
% \definecolor{DarkCharcoal}{HTML}{4D4944}
% \colorlet{Fern}{DarkFern!85!white}
% \colorlet{Charcoal}{DarkCharcoal!85!white}
% \colorlet{LightCharcoal}{Charcoal!50!white}
% \colorlet{AlertColor}{orange!80!black}
% \colorlet{DarkRed}{red!70!black}
% \colorlet{DarkBlue}{blue!70!black}
% \colorlet{DarkGreen}{green!70!black}

% Official UT colors. See style guide:
% https://brand.utexas.edu/identity/color/
\definecolor{burntorange}{HTML}{BF5700}
\definecolor{mygrey}{HTML}{333f48}

% % Use the colors:
\setbeamercolor{title}{fg=burntorange}
\setbeamercolor{structure}{fg=burntorange, bg = grey90}
\setbeamercolor{titlelike}{parent=structure}
\setbeamercolor{frametitle}{fg=burntorange}
\setbeamercolor{normal text}{fg=mygrey}
\setbeamercolor{block title}{fg=white,bg=burntorange}
\setbeamercolor{block body}{fg=mygrey,bg=grey90}

\setbeamercolor{block title example}{fg=white,bg=dodgerblue3}
\setbeamercolor{block body example}{fg=mygrey,bg=grey90}

\setbeamercolor{block title alerted}{fg=white,bg=firebrick3}
\setbeamercolor{block body alerted}{fg=mygrey,bg=grey90}


% \setbeamercolor{alerted text}{fg=AlertColor}
% \setbeamercolor{itemize item}{fg=burntorange}


%%% Local Variables:
%%% mode: latex
%%% TeX-master: "../main"
%%% End:
