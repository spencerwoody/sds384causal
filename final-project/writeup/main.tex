
% -------------------------------------------------------------------------
% Packages and margins

\documentclass[12pt]{article}
\usepackage{amsmath}
\usepackage{graphicx,psfrag,epsf}
\usepackage{enumerate}
\usepackage[round]{natbib}

\usepackage{bm}

\usepackage{authblk}            % Multiple authors
\renewcommand\Authands{ and }

\usepackage{txfonts}

\bibliographystyle{plainnat}
\usepackage{url} % not crucial - just used below for the URL 

%\pdfminorversion=4
% NOTE: To produce blinded version, replace "0" with "1" below.
\newcommand{\blind}{0}

% DON'T change margins - should be 1 inch all around.
\addtolength{\oddsidemargin}{-.5in}%
\addtolength{\evensidemargin}{-.5in}%
\addtolength{\textwidth}{1in}%
\addtolength{\textheight}{1.3in}%
\addtolength{\topmargin}{-.8in}%

% -------------------------------------------------------------------------
% Custom commands (consider different source file...)

\newcommand{\E}{\mathrm{E}}

\newcommand{\N}{\mathcal{N}}
\newcommand{\X}{\mathbf{X}}
\newcommand{\Y}{\mathbf{Y}}
\newcommand{\T}{\mathbf{T}}
\newcommand{\dd}{\mathrm{d}}
\newcommand{\bt}{\mathbf{t}}
\newcommand{\btheta}{\bm{\theta}}
\newcommand{\bpsi}{\bm{\psi}}

% -------------------------------------------------------------------------
% Begin document

\begin{document}

\def\spacingset#1{\renewcommand{\baselinestretch}%
{#1}\small\normalsize} \spacingset{1}


% -------------------------------------------------------------------------
% Title and authors

% Title for both blinded and unblinded versions...
\newcommand{\mytitle}{Literature Review for Causal Inference with
  Continuous and Multi-valued Treatments}

\if0\blind
{
  \title{\bf \mytitle}
  
  \author{Spencer Woody}
  
%   More compact way
  
%   \author[1 2]{Carlos Carvalho}

% \author[1]{Jared Murray}

% \author[2]{Spencer Woody\thanks{Corresponding author. Email to
%     \texttt{spencer.woody@utexas.edu}}}

% \affil[1]{Department of Information, Risk, and Operations Management, University~of~Texas~at~Austin}

% \affil[2]{Department of Statistics and Data Sciences, University~of~Texas~at~Austin}


  \maketitle
} \fi

\if1\blind
{
  \bigskip
  \bigskip
  \bigskip
  \begin{center}
    {\LARGE\bf mytitle}

    % Optionally, if you want the date too

    % \bigskip
    % \today
    
\end{center}
  \medskip
} \fi

% -------------------------------------------------------------------------
% Abstract and keywords

% \bigskip
% \begin{abstract}
% The text of your abstract. 200 or fewer words.
% \end{abstract}

% \noindent%
% {\it Keywords:}  3 to 6 keywords, that do not appear in the title
% \vfill

% -------------------------------------------------------------------------
% Set spacing for main text; don't change this

% \newpage
\spacingset{1.45} % DON'T change the spacing!

% -------------------------------------------------------------------------
% Main body

\section{Introduction}
\label{sec:intro}



\subsection{\cite{Hirano2004}}

Posit existence of potential outcomes $Y_i(t)$, the unit-level
dose-response function for $t \in \mathcal{T} = [t_0, t_1]$. 

Goal: average dose-response function $\mu(t) = \E [Y_i(t)]$

Assumption 1 (weak unconfoundedness): $Y(t) \perp T \mid X$ for all
$t \in \mathcal{T}$.

Generalized propensity score: Let $r(t, x)$ be the conditional density
of the treatment given the covariates:
\begin{align*}
  r(t, x) = f_{T \mid X}(t \mid x)
\end{align*}
The generalized propensity score is $R = r(T, X)$

\subsection{\cite{Imbens2000}}


\subsection{\cite{imai2004}  }

Propose a generalized propensity score function (conditional density),
and then subclassify units based on
$\hat{\bm{\theta}} = \bm{\theta}_{\hat{\bm{\psi}}}(\mathbf{X})$ which
uniquely characterizes the propensity score function.

Results:

Propensity function is a balancing score (see also Hirano \& Rubin):
\begin{align*}
  p(\T^{A} \mid X) = p\{\T^A \mid X, e(\cdot \mid X)\}
  = p\{\T^A \mid e(\cdot \mid X)\}
\end{align*}

Strong ignorability of Treatment Assignment Given the Propensity Function
\begin{align*}
  p\{Y(\bt^P)\} =
  \int p(Y(\bt) \mid \T^A = \bt^P, \btheta) p(\btheta) \dd \btheta
\end{align*}

Subclassify into $J$ groups based on
$\hat \btheta = \btheta_{\hat \bpsi}(\X)$

Use parametric model
$p_{\bm{\phi}} \{ Y(\bt^P) \mid \T^A = \bt^P\}$. Approximate
the dose response using
\begin{align*}
  p\{Y(\bt^P)\}
  &=
    \int p(Y(\bt) \mid \T^A = \bt^P, \btheta) p(\btheta) \dd \btheta
  \\
  &\approx \sum_{j=1}^{J} p_{\bm{\hat \phi}_j} \{ Y(\bt^P) \mid \T^A =
    \bt^P\} W_j
\end{align*}
with weights $W_j$

\subsection{\cite{Moodie2012}}


\subsection{\citet{pap2020}  }



% \begin{itemize}
% \item \cite{Hirano2004}
% \item \cite{Imbens2000}
% \item \cite{imai2004}
% \item \cite{Moodie2012}
% \item \citet{pap2020}
% \end{itemize}

% -------------------------------------------------------------------------
% Supplementary material



% \bigskip
% \begin{center}
% {\large\bf SUPPLEMENTARY MATERIAL}
% \end{center}

% \begin{description}

% \item[Title:] Brief description. (file type)

% \item[R-package for  MYNEW routine:] R-package \texttt{mypackage} containing code to perform the diagnostic methods described in the article. The package also contains all datasets used as examples in the article. (GNU zipped tar file)

% \item[HIV data set:] Data set used in the illustration of MYNEW method in Section~ 3.2. (.txt file)

% \end{description}


% -------------------------------------------------------------------------
% Bibliography

\pagebreak

\bibliography{main}
\end{document}


%%% Local Variables:
%%% mode: latex
%%% TeX-master: t
%%% End:
